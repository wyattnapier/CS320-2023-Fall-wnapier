\documentclass[11pt,fullpage]{article}
\usepackage{latexsym}
\oddsidemargin=-0.75cm
\evensidemargin=-0.75cm
\textwidth=17.60cm
\textheight=23cm
\topmargin=-1.50cm
\headsep=1cm

\title{\LARGE\bf BU CAS CS 320: Concepts of Programming Languages\break\break
Quiz~1}
\author{Instructor: Hongwei Xi}
\date{\today}
%
%\date{Time: 2:00-4:00pm \kern12pt Room: GCB 201 \break Date: the 23rd of June, 2023}
%
\newtheorem{lemma}{Lemma}
\newtheorem{question}[lemma]{Question}
\def\fillsquare{\kern2pt\raise0.25pt
    \hbox{$\vcenter{\hrule height0pt \hbox{\vrule width5pt height5pt} \hrule height0pt}$}}
\newenvironment{solution}{%
\vskip6pt{\em Solution\kern6pt}}{%
{\unskip\nobreak\hfill\penalty50\kern4pt\hbox{}\nobreak\hfill\fillsquare}\vskip6pt}

\begin{document}
\maketitle

\noindent
Name:\underline{\hbox to 144pt{\hss}}
\hfill
Score:\underline{\hbox to 144pt{\hss}}

\vspace{12pt}
\begin{center}
\Large
\begin{tabular}{l|r|r|p{36pt}}
No. & Points & Answer & Score \\ \hline \hline
%%
2-1. & 3 & & \\ \hline
2-2. & 3 & & \\ \hline
2-3. & 4 & & \\ \hline
2-4. & 5 & & \\ \hline
2-5. & 5 & & \\ \hline
2-6. & 10 & & \\ \hline
2-7. & 10 & & \\ \hline
2-8. & 20 & & \\ \hline

%%
Total & 60& & \\ \hline \hline

%%
\end{tabular} \\[24pt]
\end{center}
\vfill\newpage

\begin{center}
{\huge\bf No computer is allowed!} \\[6pt]
(And your phone is considered a computer.) \\[6pt]
\end{center}

\begin%
{verbatim}

Here are some definitions:

let k_combo = fun x y -> x
let s_combo = fun x y z -> x z (y z)
let church_two = fun f x -> f( f( x ) )
let church_three = fun f x -> f( f( f( x ) ) )

You can always use list constructor nil ([]) and cons (::).

\end{verbatim}


\begin%
{question}
%%
\begin%
{verbatim}
(*
Q2-1: 3 points
 
let mystery1 =
s_combo
(k_combo)
(k_combo)(fun x -> 10)(100)

What is the value of mystery1 ? Your answer:
*)
\end{verbatim}
%%
\end{question}

\vspace{12pt}

\begin%
{question}
%%
\begin%
{verbatim}
(*
Q2-2: 3 points
 
let mystery2 =
church_three(fun x -> x * x)(2)

What is the value of mystery2 ? Your answer: 
*)
\end{verbatim}
%%
\end{question}

\vspace{12pt}

\begin%
{question}
%%
\begin%
{verbatim}
(*
Q2-3: 4 points

let mystery3 =
church_three(church_two)(fun x -> x + x)(2)

What is the value of mystery3 ? Your answer:
*)
\end{verbatim}
%%
\end{question}

\vfill\newpage

\begin%
{question}
%%
\begin%
{verbatim}
(*
Q2-4: 5 points
The function list_last returns the last element of a given
list. Please give a NON-RECURSIVE implementation of list_last
based on pattern matching and list_foldleft. If the given list
is empty, raise the Empty exception

exception Empty
let list_last(xs: 'a list): 'a = ....
*)
\end{verbatim}
%%
\end{question}

\vspace{24pt}
\vspace{24pt}
\vspace{24pt}
\vspace{24pt}
\vspace{24pt}
\vspace{24pt}

\begin%
{question}
%%
\begin%
{verbatim}
(*
Q2-5: 5 points
The function list_last returns the last element of a given
list. Please give a NON-RECURSIVE implementation of list_last
based on pattern matching and list_foldright. If the given list
is empty, raise the Empty exception

exception Empty
let list_last(xs: 'a list): 'a = ....
*)
\end{verbatim}
%%
\end{question}

\vfill\newpage

\begin%
{question}
%%
\begin%
{verbatim}
(*
Q2-6: 10 points

The function list_reverse return the reverse of a given list.
Please give an implementation of list_reverse based on list_foldright
(not list_foldleft).
*)
\end{verbatim}
%%
\end{question}

\vspace{36pt}
\vspace{36pt}
\vspace{36pt}
\vspace{36pt}

\begin%
{question}
%%
\begin%
{verbatim}
(*
Q2-7: 10 points

The following implementation of list_append is not tail-recursive.
Please give an implementation of list_append that is tail-recursive.

Note that you can only use pattern matching and list_foldleft in your
implementation.
 
let rec
list_append(xs: 'a list)(ys: 'a list) =
match xs with
  [] -> ys | x1 :: xs -> x1 :: list_append(xs)(ys)
*)
\end{verbatim}
%%
\end{question}

\vfill\newpage

\begin%
{question}
%%
\begin%
{verbatim}
(*
Q2-8: 20 points

Recall the 'foreach' function and the 'get_at' function.
For instance, list_foreach(xs)(work) applies 'work' to
each element in the given list 'xs'; list_get_at(xs)(i)
returns the element at position 'i' in 'xs' if 'i' is a
valid index; otherwise the Subscript exception is raised.

Please implement in *Python* a function 'foreach_to_get_at'
that turns a 'foreach' function into a 'get_at' function.

(*
Following is the type for 'foreach_to_get_at' in ocaml:
fun foreach_to_get_at
  (foreach: ('xs, 'x0) foreach): ('xs -> int -> 'x0) = ...
*)

def foreach_to_get_at(foreach): # your implementation below
*)
\end{verbatim}
%%
\end{question}

\vfill\newpage

\end{document}
