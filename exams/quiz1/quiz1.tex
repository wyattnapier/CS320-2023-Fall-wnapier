\documentclass[11pt,fullpage]{article}
\usepackage{latexsym}
\oddsidemargin=-0.75cm
\evensidemargin=-0.75cm
\textwidth=17.60cm
\textheight=23cm
\topmargin=-1.50cm
\headsep=1cm

\title{\LARGE\bf BU CAS CS 320: Concepts of Programming Languages\break\break
Quiz~1}
\author{Instructor: Hongwei Xi}
\date{\today}
%
%\date{Time: 2:00-4:00pm \kern12pt Room: GCB 201 \break Date: the 23rd of June, 2023}
%
\newtheorem{lemma}{Lemma}
\newtheorem{question}[lemma]{Question}
\def\fillsquare{\kern2pt\raise0.25pt
    \hbox{$\vcenter{\hrule height0pt \hbox{\vrule width5pt height5pt} \hrule height0pt}$}}
\newenvironment{solution}{%
\vskip6pt{\em Solution\kern6pt}}{%
{\unskip\nobreak\hfill\penalty50\kern4pt\hbox{}\nobreak\hfill\fillsquare}\vskip6pt}

\begin{document}
\maketitle

\noindent
Name:\underline{\hbox to 144pt{\hss}}
\hfill
Score:\underline{\hbox to 144pt{\hss}}

\vspace{12pt}
\begin{center}
\Large
\begin{tabular}{l|r|r|p{36pt}}
No. & Points & Answer & Score \\ \hline \hline
%%
1-1. & 2 & & \\ \hline
1-2. & 2 & & \\ \hline
1-3. & 2 & & \\ \hline
1-4. & 2 & & \\ \hline
1-5. & 2 & & \\ \hline
1-6. & 10 & & \\ \hline
1-7. & 10 & & \\ \hline
1-8. & 20 & & \\ \hline

%%
Total & 50& & \\ \hline \hline

%%
\end{tabular} \\[24pt]
\end{center}
\vfill\newpage

\begin{center}
{\huge\bf No computer is allowed!} \\[6pt]
(And your phone is considered a computer.) \\[6pt]
\end{center}

\begin%
{question}
%%
\begin%
{verbatim}
(*

Question 1-1: 2 points

let rec f(x) = f(x)

Is 'f' tail-recursive? (1 for yes and 0 for no)

Your answer:
*)
\end{verbatim}
%%
\end{question}

\vspace{12pt}

\begin%
{question}
%%
\begin%
{verbatim}
(*

Question 1-2: 2 points

let rec f(x) = f(x+1)

Is 'f' tail-recursive? (1 for yes and 0 for no)?

Your answer:

*)
\end{verbatim}
%%
\end{question}

\vfill\newpage

\begin%
{question}
%%
\begin%
{verbatim}
(*

Question 1-3: 2 points

let rec f(x) = f(x)+1

Is 'f' tail-recursive? (1 for yes and 0 for no)?

Your answer:

*)
\end{verbatim}
%%
\end{question}

\vspace{12pt}

\begin%
{question}
%%
\begin%
{verbatim}
(*

Question 1-4: 2 points

let rec f(x) =
if x > 0 then f(f(x)) else f(f(f(x)))

How many tail-recursive calls in the definition of 'f'?

Your answer:

*)
\end{verbatim}
%%
\end{question}

\vspace{12pt}

\begin%
{question}
%%
\begin%
{verbatim}
(*

Question 1-5: 2 points

let
rec f(x) = f(g(f(x))) + 1
and g(y) = f(g(f(g(f(g(x))))))

How many (mutual) tail-recursive calls in the definition of 'g'?

Your answer:

*)
\end{verbatim}
%%
\end{question}

\vfill\newpage

\begin%
{question}
%%
\begin%
{verbatim}
(*

Question 1-6: 10 points

let pp x y = fun f -> f(x,y)

let ff = (* WRITE YOUR CODE *)

(*
Given an implementation of ff that
makes the following assertions pass
*)

let () = assert (pp 200 100 ff = 100)
let () = assert (pp 100 200 ff = -100)

Please present your code as follows:
*)
\end{verbatim}
%%
\end{question}

\vfill\newpage

\begin%
{question}
%%
\begin%
{verbatim}
(*
Question 1-7: 10 points

Given the following snippet, implement the test
function so that isPrime returns true for prime
number inputs and false otherwise.

let isPrime(n) =
let test(i:int): bool = (* YOUR CODE *)
in
  if n < 2 then false else int1_forall(n)(test)

Please present your code as follows: 
*)
\end{verbatim}
%%
\end{question}

\vfill\newpage

\begin%
{question}
%%
\begin%
{verbatim}
(*
Question 1-8: 20 points
Please give a NON-RECURSIVE implementation of sort5
that takes 5 integers and returns a tuple that consists
exactly of the 5 given integers ordered increasingly

let sort5: int*int*int*int*int -> int*int*int*int*int =
  (* YOUR CODE *)

For instance, sort5(1, 2, 1, 2, 1) = (1, 1, 1, 2, 2)
For instance, sort5(1, 3, 4, 5, 2) = (1, 2, 3, 4, 5)
For instance, sort5(1, 3, 5, 4, 2) = (1, 2, 3, 4, 5)

You can implement your own helper functions as long as
you do not make use of recursion.

Note that we are not looking for a solution solely based
on a very large embedded if-then-else expression here.

Please present your code as follows:
*)
\end{verbatim}
%%
\end{question}

\vfill\newpage

\end{document}
